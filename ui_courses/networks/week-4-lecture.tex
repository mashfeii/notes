\documentclass{article}
\usepackage[english]{babel}
\usepackage[letterpaper,top=2cm,bottom=2cm,left=2.5cm,right=2.5cm,marginparwidth=1.25cm]{geometry}

\usepackage[leqno]{amsmath}
\usepackage{enumitem, nccmath,lipsum,amssymb,xcolor,xparse,listings, blindtext, hyperref, float, booktabs, tabularx}
\usepackage[most]{tcolorbox}

\NewDocumentCommand{\codeword}{v}{
\texttt{\textcolor{blue}{#1}}
}

\lstset{language=C, keywordstyle={\bfseries \color{blue}}}

\NewDocumentCommand{\mynote}{+O{}+m}{%
  \begingroup
  \tcbset{%
    noteshift/.store in=\mynote@shift,
    noteshift=1.5cm
  }
  \begin{tcolorbox}[nobeforeafter,
    enhanced,
    sharp corners,
    toprule=0.5pt,
    bottomrule=0.5pt,
    leftrule=0pt,
    rightrule=0pt,
    colback=green!10,
    #1,
    left skip=\mynote@shift,
    right skip=\mynote@shift,
    overlay={\node[right] (mynotenode) at ([xshift=-\mynote@shift]frame.west) {\textbf{Note:}} ;},
    ]
    #2
  \end{tcolorbox}
  \endgroup
  }
\makeatother

\newcommand{\mytext}[1]% #1 = same as intertext
{&\parbox{0.9\textwidth}{\rule{0pt}{.5\baselineskip}\\
\textrm{#1}\\
\rule{0pt}{.5\baselineskip}}&\\}

\newcounter{exercise}
\newcounter{problem}[exercise]
\newcommand{\myitem}{\stepcounter{problem}\tag*{\alph{problem})}}

\title{Lecture Week 4.\\ Electronic Mail Timeline}
\author{Mashenkov Timofei CBS-02 \\ \href{mailto:t.mashenkov@innopolis.university}{t.mashenkov@innopolis.university}}
\begin{document}
\maketitle{}

\section*{Electronic Mail Timeline}

\begin{itemize}
	\item \textbf{1961. MIT Compatible Time-Sharing System (CTSS):} \\
	      The first general time-sharing OS to support simultaneous users to MIT's IBM 7094 computer from remote terminals (typewriters)
	\item \textbf{1965. "MAIL" command in CTSS operating system:} \\
	      Communication between remote clients by using a shared text file stored on a central hard disk
	\item \textbf{1971. ARPANET's network email system by Ray Tomlinson:} \\
	      Based on FTP for files transfer, limited to an internal network; \\
	      Introduced electronic mail destination addresses of type "username@name of computer"
	\item \textbf{1972. The first email management tool by Larry Roberts}\\
	      (also in the scope of ARPANET project): \\
	      To reply, forward, select, and other operations with emails
	\item \textbf{1976. First attempts of electronic mail sending to external networks}
	\item \textbf{1981. SMTP protocol release} \\
	      (implemented on the ARPANET network in 1983): \\
	      Communication over the Internet between different computers spreads
	\item \textbf{1980s. Multiple LAN email systems emerged}
	\item \textbf{1988, 1989. Eudora by Steve Dorner and Lotus Notes applications for email management:} \\
	      Email management application with grahpical user interfaces
	\item \textbf{1993-1996. AOL, Echomail, Hotmail, Yahoo Mail services} \\
	      Email services directly in Internet browsers
	\item \textbf{1996. Internet Mail and News 1.0 service by Microsoft:} \\
	      Later renamed into Outlook
\end{itemize}

\section*{Email Address Format}

Was specified by IETF organization in RFC 5322 and 6854, examples:

\begin{itemize}
	\item \codeword{ithelp@university.innopolis.ru}
	\item \codeword{"it help"@university.innopolis.ru}
\end{itemize}

\mynote{$<local\_part> @ <domain\_name>$}

\begin{itemize}
  \item \textbf{<local_part>}:  
  \item \textbf{<domain_name>}: Hostname of an email server, must compy to DNS naming standards, like \textit{unviersity.innopolis.ru}  
\end{itemize}



\end{document}
