\documentclass{article}
\usepackage[english]{babel}
\usepackage[letterpaper,top=2cm,bottom=2cm,left=2.5cm,right=2.5cm,marginparwidth=1.25cm]{geometry}

\usepackage[leqno]{amsmath}
\usepackage{enumitem, nccmath,lipsum,amssymb,xcolor,xparse,listings, blindtext}
\usepackage[most]{tcolorbox}

\NewDocumentCommand{\codeword}{v}{
\texttt{\textcolor{blue}{#1}}
}

\lstset{language=C, keywordstyle={\bfseries \color{blue}}}

\NewDocumentCommand{\mynote}{+O{}+m}{%
  \begingroup
  \tcbset{%
    noteshift/.store in=\mynote@shift,
    noteshift=1.5cm
  }
  \begin{tcolorbox}[nobeforeafter,
    enhanced,
    sharp corners,
    toprule=0.5pt,
    bottomrule=0.5pt,
    leftrule=0pt,
    rightrule=0pt,
    colback=green!10,
    #1,
    left skip=\mynote@shift,
    right skip=\mynote@shift,
    overlay={\node[right] (mynotenode) at ([xshift=-\mynote@shift]frame.west) {\textbf{Note:}} ;},
    ]
    #2
  \end{tcolorbox}
  \endgroup
  }
\makeatother

\newcommand{\mytext}[1]% #1 = same as intertext
{&\parbox{0.9\textwidth}{\rule{0pt}{.5\baselineskip}\\
\textrm{#1}\\
\rule{0pt}{.5\baselineskip}}&\\}

\newcounter{exercise}
\newcounter{problem}[exercise]
\newcommand{\myitem}{\stepcounter{problem}\tag*{\alph{problem})}}

\title{Lab 2: The filesystem, command line, and file manipulation}
\author{Mashenkov Timofei}
\begin{document}
\maketitle{}

\section*{File system}

When one deletes file using OS features like \texttt{rm}, it deletes file from filsystem, but data still remains in the
storage.
It is made for efficciency, since deleting data from storage is complex and time-consuming operation.

\section*{Commands}

\texttt{chmod} - used for changing file permissions:

\begin{itemize}
	\item 4 - read
	\item 2 - write
	\item 1 - execute
\end{itemize}

Command's output can be transferred to various places using following operations

\begin{itemize}
	\item \textbf{redirection}
	      \begin{itemize}
		      \item \codeword{echo "message" > filename}

		            Used to fill file with output
		      \item \codeword{echo "message" >> filename}

		            Used to append output into the file
		      \item \codeword{cat nofile 2> filename}

		            Used to redirect stderr (second out stream) to file

		      \item \codeword{cat nofile > allout 2>&1}

		            Used to redirect stderr to stdout

	      \end{itemize}
  \item \textbf{pipe}
        \begin{itemize}
          \item \codeword{echo "message" | cat > filename}

            Pipe is used where first result of command is used as input for the second command
        \end{itemize}
\end{itemize}


\end{document}
