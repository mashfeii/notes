\documentclass{article}
\usepackage[english]{babel}
\usepackage[letterpaper,top=2cm,bottom=2cm,left=2.5cm,right=2.5cm,marginparwidth=1.25cm]{geometry}

\usepackage[leqno]{amsmath}
\usepackage{enumitem, nccmath,lipsum,amssymb,xcolor,xparse,listings, blindtext, tikz, pgfplots}
\usepackage[most]{tcolorbox}

\NewDocumentCommand{\codeword}{v}{
\texttt{\textcolor{blue}{#1}}
}

\lstset{language=C, keywordstyle={\bfseries \color{blue}}}

\newcommand{\RN}[1]{%
  \textup{\uppercase\expandafter{\romannumeral#1}}%
}

\NewDocumentCommand{\mynote}{+O{}+m}{%
  \begingroup
  \tcbset{%
    noteshift/.store in=\mynote@shift,
    noteshift=1.5cm
  }
  \begin{tcolorbox}[nobeforeafter,
    enhanced,
    sharp corners,
    toprule=0.5pt,
    bottomrule=0.5pt,
    leftrule=0pt,
    rightrule=0pt,
    colback=green!10,
    #1,
    left skip=\mynote@shift,
    right skip=\mynote@shift,
    overlay={\node[right] (mynotenode) at ([xshift=-\mynote@shift]frame.west) {\textbf{Note:}} ;},
    ]
    #2
  \end{tcolorbox}
  \endgroup
  }
\makeatother

\newcommand{\mytext}[1]% #1 = same as intertext
{&\parbox{0.9\textwidth}{\rule{0pt}{.5\baselineskip}\\
\textrm{#1}\\
\rule{0pt}{.5\baselineskip}}&\\}

\newcounter{exercise}
\newcounter{problem}[exercise]
\newcommand{\myitem}{\stepcounter{problem}\tag*{\alph{problem})}}

\title{Differential Equations: Lab.}
\author{Mashenkov Timofei}
\begin{document}
\maketitle{}

\section*{Week 11.}

\subsection*{Task 1021.}

\addtolength{\jot}{1pt}
\begin{fleqn}[1\parindent]
  \begin{gather*}
    \begin{cases}
      \dot{x}=2x+y^2-1\\ 
      \dot{y}=6x-y^2+1
    \end{cases} \rightarrow 
    \frac{dy}{dx} = \frac{0}{0} \\ 
    1)\ \begin{cases}
      2x+y^2-1=0 \\ 
      6x-y^2+1=0
    \end{cases} \rightarrow 
    \begin{matrix}
      8x = 0 & 0-y^2+1=0 \\ 
      x = 0 & y = \pm1
    \end{matrix} \\ 
    2)\ \begin{cases}
      u=x-x_0 \\ 
      v = y-y_0
    \end{cases} \\ 
    3)\ (x,y) = f(x_0,y_0)+f'_x(x_0,y_0)(x-x_0)+f'_y(x_0,y_0)(y-y_0) \\
    4)\ \begin{cases}
      \dot{u}=au+bv \\ 
      \dot{v}=du+fv 
    \end{cases} \\ 
    5) \\ 
    M_0(0;1) \\ 
    \begin{cases}
      u=x-x_0 \\ 
      v=y-y_0
    \end{cases} \\ 
    M(0;-1) \\ 
    \begin{cases}
      u=x \\ 
      v=y-1
    \end{cases} \rightarrow
    \begin{matrix}
      \dot{u}_t=\dot{x}_t\\ 
      \dot{v}_t=\dot{y}_t
    \end{matrix} \\ 
    \dot{u} = 2u+(v+1)-1=2u+v^2+2v+1-1 \\ 
    \dot{v} = 6u-(v^2+2v+1)+1 \Rightarrow 
    \begin{cases}
      \dot{u}=2u+2v+v^2 \\ 
      \dot{v}=6u-2v-v^2
    \end{cases} \\ 
    f(v) = v^2 = f(0)+f'_v(0)v=0 \\ 
    v^2 = 0;\ 2v' = 0 \\ 
    f(u,v)=f(0,0)+f'_v(u,v)\cdot v+f'_u(u,v)\cdot u \\
    \begin{cases}
      \dot{u} = 2u+2v \\ 
      \dot{v} = 6u-2v
    \end{cases} \\ 
    \begin{vmatrix}
      2-\lambda & 2 \\ 
      6 & -2-\lambda
    \end{vmatrix} = 0 \\ 
    (\lambda-2)(\lambda+2)-12=0 \\ 
    \lambda^2=16 \\ 
    \lambda = \pm 4 \\ 
  \end{gather*}
\end{fleqn}

\addtolength{\jot}{1pt}
\begin{fleqn}[1\parindent]
  \begin{gather*}
    6)\ \text{Saddle point} \\
    \lambda_1 = 4: \\ 
    \begin{bmatrix}
      2-4 & 2 \\ 
      6 & -2-4
    \end{bmatrix}\begin{pmatrix}
      u \\ v
    \end{pmatrix} = 0 \\ 
    u = v \rightarrow \begin{pmatrix}
      1 \\ 1
    \end{pmatrix} \\ 
    \lambda_2 = -4: \\
    \begin{bmatrix}
      6 & 2 \\ 
      6 & 2
    \end{bmatrix}
    \begin{pmatrix}
      u \\ v
    \end{pmatrix} 
    = 0 \rightarrow \begin{pmatrix}
      1 \\ -3
    \end{pmatrix} \\ 
    v = -3u \\ 
    \begin{cases}
      \dot{u}=2u+2v \\ 
      \dot{v}=6u-2v
    \end{cases} \\
    \text{Repeat for } M_1(0;-1): \\ 
    \begin{cases}
      u=x \\ 
      v=y+1
    \end{cases} \\
    \dot{u}=2u+(v-2v+1)-1 \\ 
    \dot{u}=2u-2v \\ 
    \dot{v}=6u-(v^2-2v+1)-1 \\ 
    \dot{v}=6u+2v \\ 
    \begin{cases}
      \dot{u}=2u-2v \\ 
      \dot{v}=6u+2v
    \end{cases} \\ 
    \begin{vmatrix}
      2-\lambda & -2 \\ 
      6 & 2-\lambda
    \end{vmatrix} = 0 \\ 
    (2-\lambda)^2+12=0 \\ 
    2-\lambda = \pm\sqrt{-12} \\ 
    \lambda = 2 \pm \sqrt{-12} \\
    \text {Focus, since Real part $>$ 0} \\ 
  \end{gather*}
\end{fleqn}

\subsection*{Task 920.}

\addtolength{\jot}{1pt}
\begin{fleqn}[1\parindent]
  \begin{gather*}
    \begin{cases}
      \dot{x}=-e^x+e^y \\ 
      \dot{y}=\sqrt{3x+y^2}-2 
    \end{cases} \\ 
    \begin{cases}
      \dot{u}=e(e^v-e^u) \\ 
      \dot{v}=\sqrt{v^2+2v+3u+4}-2
    \end{cases} \\ 
    f(i,v)=e^u-e^0+e^v=1+v;\ e^u \approx 1+u \\ 
    \dot{u}=-e^{u+1}+e^{v+1}=e(e^v-e^u) \\ 
    \dot{v}=\sqrt{3(u+1)+(v+1)^2}-2 \\
    \sqrt{v^2+2v+3u+4}=2+\frac{1}{\sqrt{4}}(2v+2)v+\frac{1}{2\sqrt{4}}\cdot 3 \cdot u = 
    2+\frac{1}{2}v+\frac{3}{4}u \\ 
    \begin{cases}
      \dot{u}=e(1+v-(1+u)) \\ 
      \dot{v}=2+\frac{v}{2}+\frac{3}{4}u-2
    \end{cases} \\ 
    \begin{cases}
      \dot{u}=e(v-u) \\ 
      \dot{v} =\frac{v}{2}+\frac{3}{4}u
    \end{cases} \\ 
    (e+\lambda)(\lambda-\frac{1}{2})-\frac{3}{4}e=0 \\ 
    e\lambda-\frac{1}{2}e+\lambda^2-\frac{1}{2}\lambda-\frac{3}{4}e=0 \\
    \lambda^2+\lambda(e - \frac{1}{2})-\frac{5}{4}e=0 \\ 
    \lambda_{1,2}=\frac{-(e-\frac{1}{2})\pm\sqrt{(e-\frac{1}{2})^2+5e}}{2} \\
    \boxed{\lambda_1\cdot\lambda_2 < 0 \rightarrow \text{ Saddle point}}
  \end{gather*}
\end{fleqn}

\addtolength{\jot}{1pt}
\begin{fleqn}[1\parindent]
  \begin{gather*}
    M_0(1;1) \rightarrow \boxed{\text{Unstable Saddle}} \\
    M_1(-4;-4): \\ 
    \lambda^2 +\lambda(2+\frac{1}{e^4})+\frac{5}{4e^4}=0 \\ 
    \lambda_{1,2} < 0 \rightarrow \boxed{\text{Stable knot}}
  \end{gather*}
\end{fleqn}

\subsection*{Task 1003.}

\addtolength{\jot}{1pt}
\begin{fleqn}[1\parindent]
  \begin{gather*}
    \ddot{x}-x+x^2=0 \\ 
    \dot{x}=y \\ 
    \begin{cases}
      \dot{x}=y \\ 
      \dot{y}=x-x^2
    \end{cases} \\
    M_0(0;0),\ M_1(1;0) \\
  \end{gather*}
\end{fleqn}

\subsection*{Task 1009.}

\addtolength{\jot}{1pt}
\begin{fleqn}[1\parindent]
  \begin{gather*}
    \ddot{x}-\sin{x}=0 \\ 
    \begin{cases}
      \dot{x}=y=0 \\ 
      \dot{y}=\sin{x}=0 \leftarrow x = \pi k
    \end{cases} \\ 
  \end{gather*}
\end{fleqn}

\end{document}
