\documentclass{article}
\usepackage[english]{babel}
\usepackage[letterpaper,top=2cm,bottom=2cm,left=2.5cm,right=2.5cm,marginparwidth=1.25cm]{geometry}

\usepackage[leqno]{amsmath}
\usepackage{enumitem, nccmath,lipsum,amssymb,xcolor,xparse,listings, blindtext,array}
\usepackage[most]{tcolorbox}

\NewDocumentCommand{\codeword}{v}{
\texttt{\textcolor{blue}{#1}}
}

\lstset{language=C, keywordstyle={\bfseries \color{blue}}}

\NewDocumentCommand{\mynote}{+O{}+m}{%
  \begingroup
  \tcbset{%
    noteshift/.store in=\mynote@shift,
    noteshift=1.5cm
  }
  \begin{tcolorbox}[nobeforeafter,
    enhanced,
    sharp corners,
    toprule=0.5pt,
    bottomrule=0.5pt,
    leftrule=0pt,
    rightrule=0pt,
    colback=green!10,
    #1,
    left skip=\mynote@shift,
    right skip=\mynote@shift,
    overlay={\node[right] (mynotenode) at ([xshift=-\mynote@shift]frame.west) {\textbf{Note:}} ;},
    ]
    #2
  \end{tcolorbox}
  \endgroup
  }
\makeatother

\newcommand{\mytext}[1]% #1 = same as intertext
{&\parbox{0.9\textwidth}{\rule{0pt}{.5\baselineskip}\\
\textrm{#1}\\
\rule{0pt}{.5\baselineskip}}&\\}

\newcounter{exercise}
\newcounter{problem}[exercise]
\newcommand{\myitem}{\stepcounter{problem}\tag*{\alph{problem})}}

\title{Operating Systems: Input/Output}
\author{Mashenkov Timofei}
\begin{document}
\maketitle{}

\section*{RAID Systems}

\subsection*{General Information}

\begin{fleqn}
  \begin{tabular}{| m{1cm} | m{12cm} |}
    \hline
    RAID & Description  \\ 
    \hline
    0 & Stripping only, no mirroring, no parity checks \\ 
    \hline
    1 & Mirroring with or without striping; no parity checks \\ 
    \hline
    2 & Bit-level striping with Hamming-code based parity; one byte is split into two parts, three parity bits are
    added; resulting bits are written to seven disks, one bit per disk (all disks' rotation is synchronized) \\
    \hline
    3 & Bit-level striping with a parity bit computed for each word and stored on dedicated party drive \\ 
    \hline
    4 & Strip-for-strip parity: all the strips are XOR'ed together, resulting in a parity strip \\ 
    \hline
    5 & The same as RAID 4, except that parity bits are distributed across all the drives in round-robin fashion \\
    \hline
  \end{tabular}
\end{fleqn}

\subsection*{Reliability}

\begin{fleqn}
  \begin{tabular}{| m{1cm} | m{12cm} |}
    \hline
    RAID & Reliability \\ 
    \hline
    0 & No reliability \\ 
    \hline
    1 & Can survive one disk crash \\ 
    \hline
    2 & Can survive one disk crash; a single random bit error in a word can be detected and corrected \\
    \hline
    3 & Can survive one disk crash; a single random bit error in a word can be detected \\ 
    \hline
    4 & Can survive one disk crash; a single random bit error in a word can be detected \\ 
    \hline
    5 & Can survive one disk crash; a single random bit error in a word can be detected \\
    \hline
  \end{tabular}
\end{fleqn}

\end{document}
