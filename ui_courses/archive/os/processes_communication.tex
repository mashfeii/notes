\documentclass{article}
\usepackage[english]{babel}
\usepackage[letterpaper,top=2cm,bottom=2cm,left=2.5cm,right=2.5cm,marginparwidth=1.25cm]{geometry}

\usepackage[leqno]{amsmath}
\usepackage{enumitem, nccmath,lipsum,amssymb,xcolor,xparse,listings, blindtext}
\usepackage[most]{tcolorbox}

\NewDocumentCommand{\codeword}{v}{
\texttt{\textcolor{blue}{#1}}
}

\definecolor{codegreen}{rgb}{0,0.6,0}
\definecolor{codegray}{rgb}{0.5,0.5,0.5}
\definecolor{codepurple}{rgb}{0.58,0,0.82}
\definecolor{backcolour}{rgb}{0.95,0.95,0.92}

\lstdefinestyle{mystyle}{
  backgroundcolor=\color{backcolour},   
  commentstyle=\color{codegreen},
  keywordstyle=\color{magenta},
  numberstyle=\tiny\color{codegray},
  stringstyle=\color{codepurple},
  basicstyle=\ttfamily\footnotesize,
  breakatwhitespace=true,         
  breaklines=true,                 
  captionpos=b,                    
  keepspaces=false,                 
  numbers=left,                    
  numbersep=3pt,                  
  showspaces=false,                
  showstringspaces=false,
  showtabs=false,                  
  tabsize=2
}

\lstset{style=mystyle}

\NewDocumentCommand{\mynote}{+O{}+m}{%
  \begingroup
  \tcbset{%
    noteshift/.store in=\mynote@shift,
    noteshift=1.5cm
  }
  \begin{tcolorbox}[nobeforeafter,
    enhanced,
    sharp corners,
    toprule=0.5pt,
    bottomrule=0.5pt,
    leftrule=0pt,
    rightrule=0pt,
    colback=green!10,
    #1,
    left skip=\mynote@shift,
    right skip=\mynote@shift,
    overlay={\node[right] (mynotenode) at ([xshift=-\mynote@shift]frame.west) {\textbf{Note:}} ;},
    ]
    #2
  \end{tcolorbox}
  \endgroup
  }
\makeatother

\newcommand{\mytext}[1]% #1 = same as intertext
{&\parbox{0.9\textwidth}{\rule{0pt}{.5\baselineskip}\\
\textrm{#1}\\
\rule{0pt}{.5\baselineskip}}&\\}

\newcounter{exercise}
\newcounter{problem}[exercise]
\newcommand{\myitem}{\stepcounter{problem}\tag*{\alph{problem})}}

\title{Operating Systems. Processes Communication}
\author{Mashenkov Timofei}
\begin{document}
\maketitle{}

\section*{Race Conditions}

Situation when several processes try to access/modify shared resources at the same time.

\subsection*{Avoiding Race Conditions}

\textbf{Critical Regions}

A way to prohibit more than one process from reading and writing the shared data at the same time → \textbf{mutual
exclusion}

\begin{itemize}
  \item No two processes may be simultaneously inside their critical regions.
  \item No assumptions may be made about speed or the number of CPUs.
  \item No process running outside its critical region may block other processes.
  \item No process should have to wait forever to enter its crtical region.
\end{itemize}

How to implement mutual exclusion?

\textbf{Disabling Interrupts}

\textbf{Lock Variables}

A possible software solution, but this idea contains fatal flaw that is called \textbf{Deadlock}:

\begin{itemize}
  \item To have a shared variable (lock) that is initially 0
  \item A process entering its critical region tests the lock first
  \item Lock == 0? The process sets it to 1 and enters the critical region
  \item Lock == 1? The process waits until the lock is set to 0
\end{itemize}

\textbf{Strict Alternation}

\begin{itemize}
  \item To have an integer variable (turn) that is initially 0
  \item Process 0 finds it to be 0, and enters its critical region
  \item Process 1 also finds 
\end{itemize}

\textbf{The Producer-Consumer Problem}

\begin{itemize}
  \item To keep track of the number of items in the buffer, we will need a variable, \textbf{\textit{count}}
  \item The producer's code will first test to see if count is N (the maximum number of items in the buffer) 
  \item If it is, the producer will go to sleep; if it is not, the producer will add an item and increment count 
  \item Consumer first tests \textbf{\textit{count}} to see if it is 0
  \item If it is, go to sleep; if it is nonzero, remove an item and decrement the counter 
  \item Each of the processes also tests to see if the other should be awakened, and if so, wakes it up
\end{itemize}

\textbf{Semaphores}

\begin{itemize}
  \item Proposed by Dijkstra In 1965 
  \item an integer variable to count the number of wake-ups saved for future use - a \textbf{semaphore}
  \item has a positive value if one or more wake-ups were pending, 0 otherwise
  \item two operations: \textbf{down} and \textbf{up} (generalizations of sleep a \textbf{wake-up}, respectively)
  \item the down operation checks if the value is greater than 0. If so, it decrements the value (i.e., uses up one
    stored wake-up) and just continues
\end{itemize}

\begin{lstlisting}[language=C]
#define N 100 // number of slots in the buffer
typedef int semaphore; // semaphores are a special kind of int 
semahpore mutex = 1; // controls access to critical region
semaphore empty = N; // counts empty buffer slots
semphore full = 0; // count full buffer slots

void producer(void) {
  int item;

  while (TRUE) {
    item = produce_item(); // generate something to put in buffer
    down(&empty); // decrement empty count
    down(&empty); // enter critical region
    insert_item(item); // put new item in buffer
    up(&mutex); // leave critical region
    up(&full); // counts full buffer slots
  }
}

void consumer(void) {
  int item;
  while (TRUE) {
    down(&full); // decrement full count
    down(&mutex); // enter critical reigon
    item = remove_item(); // take item from buffer
    up(&mutex); // leave critical region
    up(&empty); // increment count of empty slots
    consume_item(item); // do something with the item
  }
}
\end{lstlisting}

\textbf{Monitor}

\begin{itemize}
  \item A monitor is a higher-level synchronization primitive - it is a collection of procedures, variables, and data
    structures that are all grouped together in a special kind of 
\end{itemize}

\begin{lstlisting}[language=C]
monitor ProducerConsumer
  condition full, empty;
  integer count;

  procedure insert(item: integer);
  begin
    if count = N then wait(full);
    insert_item(item);
    count := count + 1;
    if count = 1 then signal(empty);
  end;

  procedure remove: integer;
  begin
    if count = 0 then wait(empty);
    remove = remove_item;
    count := count - 1;
    if count = N - 1 then signal(full);
  end;

  count := 0;
end monitor;
\end{lstlisting}

\textbf{Message Passing}

Method of IPC that uses two primitives, send and receive, which, like semaphores and unlike monitors, are system calls
rather than language constructs.
They can easily be put into library procedures, such as:
\begin{itemize}
  \item send(destination, \&message);
  \item receive(source, \&message);
\end{itemize}

\begin{lstlisting}[language=C]
send(int fd, char* buf, int count);
write(int fd, char* buf, int count);

fd = open("file", "r"); // file descriptor

int arr[2];
pipe(arr);
int fd_int = arr[0], fd_out = arr[1];

if (fork()) { // fork() > 0 -> parent process
  write(fd_out, "hi, child", strlen("hi, child"));
} else { // child process
  read(fd_in, buf, strlen("hi, child"));
}
\end{lstlisting}

\begin{lstlisting}[language=C]
#include <stdlib.h>
#include <unistd.h>

int fields[2];
const int BSIZE = 100;
char buf[BSIZE];
ssize_t nbytes;
int status; 

status = pipe(fields);
if (status == - 1) {
  // an error occurred
}
switch(fork()) {
case -1: // handle error
  break;
case 0: // child 
  close(fields[0]); // write end is unused
  nbytes = read(fields[0], buf, BSIZE); // get data from pipe
  close(fields[0]); // finished with pipe
  exit(EXIT_SUCCESS);
default: // parent
  close(fields[0]); // read end is unused
  write(fields[1], "Hello world\n", 12); // write data on pipe
  close(fields[0]); // child will see EOFJ
  exit(EXIT_SUCCESS);
}
\end{lstlisting}

\end{document}
