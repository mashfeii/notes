\documentclass{article}
\usepackage[english]{babel}
\usepackage[letterpaper,top=2cm,bottom=2cm,left=2.5cm,right=2.5cm,marginparwidth=1.25cm]{geometry}

\usepackage[leqno]{amsmath}
\usepackage{enumitem, nccmath,lipsum,amssymb,xcolor,xparse,listings, blindtext,array}
\usepackage[most]{tcolorbox}

\NewDocumentCommand{\codeword}{v}{
\texttt{\textcolor{blue}{#1}}
}

\lstset{language=C, keywordstyle={\bfseries \color{blue}}}

\NewDocumentCommand{\mynote}{+O{}+m}{%
  \begingroup
  \tcbset{%
    noteshift/.store in=\mynote@shift,
    noteshift=1.5cm
  }
  \begin{tcolorbox}[nobeforeafter,
    enhanced,
    sharp corners,
    toprule=0.5pt,
    bottomrule=0.5pt,
    leftrule=0pt,
    rightrule=0pt,
    colback=green!10,
    #1,
    left skip=\mynote@shift,
    right skip=\mynote@shift,
    overlay={\node[right] (mynotenode) at ([xshift=-\mynote@shift]frame.west) {\textbf{Note:}} ;},
    ]
    #2
  \end{tcolorbox}
  \endgroup
  }
\makeatother

\newcommand{\mytext}[1]% #1 = same as intertext
{&\parbox{0.9\textwidth}{\rule{0pt}{.5\baselineskip}\\
\textrm{#1}\\
\rule{0pt}{.5\baselineskip}}&\\}

\newcounter{exercise}
\newcounter{problem}[exercise]
\newcommand{\myitem}{\stepcounter{problem}\tag*{\alph{problem})}}

\title{Operating Systems: Final Exam}
\author{Mashenkov Timofei}
\begin{document}
\maketitle{}

\section*{Lecture 1. General.}
\noindent

\subsection*{OS Zoo.}
\noindent

\begin{itemize}
  \item Mainframe: large room-stations with a lot of I/O work: batch, transaction-processing, time-sharing
  \item Server: FreeBSD, Windows Server, Solaris
  \item Multiprocessor: Linux, Windows
  \item HandHeld: iOS, Android
  \item Embedded: Linux, QNX, VxWorks
  \item Sensor-Node: TinyOS 
  \item Real-Time: QNX
\end{itemize}

\section*{Processes and Threads}
\noindent

A process is an \textbf{abstraction} of a running program.
\textbf{Pseudo parallelism} is an illusion of real parallelism, when processor switches between different processes.

\textbf{Conceptually} each process has its own CPU, feeling line "the main one", but in reality CPU switches between
them, so-called \textbf{multiprogramming}.
\newline

\mynote{A program is like recipe, itself it's doing nothing. But process is a running instance of program, including
program, input, output, and a state.}

Process and its children create \textbf{process hierarchy}. In UNIX, they together form a \textbf{process group}.

The OS maintains a table of process and their state, also called process control block.

Each entity contains:

\begin{itemize}
  \item its program counter
  \item stack pointer
  \item memory allocation 
  \item the status of its open files
  \item its accounting and scheduling info, etc.
\end{itemize}

\subsection*{Interrupts.}
\noindent

Interrupt is a \textbf{signal} that a process is to be \textbf{rescheduled}, the following is proceed:

\begin{enumerate}
  \item Hardware stacks program counter, etc. 
  \item Hardware loads new programs from interrupt vector.
  \item Assembly language procedure saved registers. 
  \item Assembly language procedure sets up new stack.
  \item C interrupts service runs.
  \item Scheduler decides which process to run next.
  \item C procedure returns to assembly code.
  \item Assembly language procedure loads new program.
\end{enumerate}

\subsection*{Threads.}
\noindent

\begin{center}
  \begin{tabular}{| m{4.2cm} | m{3cm} |}
    \hline \\[-1em]
    \textbf{Per process items} & \textbf{Per thread items} \\
    \hline \\[-1em]
    Address space & Program counter \\ 
    \hline \\[-1em]
    Global variables & Registers \\ 
    \hline \\[-1em]
    Open files & Stack \\ 
    \hline \\[-1em]
    Child processes & State \\ 
    \hline \\[-1em]
    Pending alarms & \\ 
    \hline \\[-1em]
    Signals and signal handlers & \\ 
    \hline \\[-1em]
    Accounting information &  \\ 
    \hline
  \end{tabular}
\end{center}

\subsection*{CPU Utilization (Degree of Multiprogramming).}
\noindent

It is the function showing the number of processes in memory $CPU\ Utilization = 1 - p^n$.

\section*{Interprocess Communication.}
\noindent

\textbf{InterProcess Communication} or \textbf{IPC} should be carried out in a well-structured way not using interrupts.

\subsection*{Avoiding Race Conditions.}
\noindent

\begin{itemize}
  \item Critical Regions
  \item Mutual Exclusion and Busy Waiting
  \item Sleep and Wake-up 
  \item Semaphores 
  \item Mutexes 
  \item Monitors 
  \item Message Passing
  \item Barriers 
  \item Avoiding Locks: Read-Copy-Update
\end{itemize}

\subsection*{Critical Regions.}
\noindent

A way to prohibit more than one process read or write shared data at the same time is called \textbf{mutual exclusion}.

Place where the shared memory is accessed are called \textbf{critical region} or \textbf{critical section}.

Requirements to implement critical regions:

\begin{itemize}
  \item No two processes may be simultaneously inside their critical regions.
  \item No assumptions may be made about speed or the number of CPUs.
  \item No process running outside its critical region may block other processes.
  \item No process should have to wait forever to enter its critical region.
\end{itemize}

\textbf{Disabling Interrupts.}

Each process disable its interrupts entering the critical region and re-enable them before leaving.

Issues:

\begin{itemize}
  \item if process do not re-enable interrupts it will stop the system
  \item do not work in multiprocessor system, since interrupts stop for single CPU
\end{itemize}

\textbf{Lock Variables.}

The idea is to introduce new logic variables and change it value on each entering/leaving. Contains the same fatal flaw
as in previous case.

\subsection*{Strict Alternation.}
\noindent

Implement complementary logic using \textbf{turn} variable.

Since method contains tight loop, it calls \textbf{busy waiting}, and the variable itself \textbf{spin-lock}.

\subsection*{Sleep and Wake-up.}
\noindent

Peterson's solution and solutions using TSL (Test Set Lock) or XCHG (Exchange Register) have \textbf{busy waiting}.

Producer-consumer problem, fixed shared buffer, complement variable \textbf{count} to compare the total number to \textit{N} or \textit{0}.

\subsection*{Semaphores.}
\noindent

\begin{itemize}
  \item Proposed by Dijkstra In 1965 
  \item an integer variable to count the number of wake-ups saved for future use - a \textbf{semaphore}
  \item has a positive value if one or more wake-ups were pending, 0 otherwise
  \item two operations: \textbf{down} and \textbf{up} (generalizations of sleep a \textbf{wake-up}, respectively)
  \item the down operation checks if the value is greater than 0. If so, it decrements the value (i.e., uses up one
    stored wake-up) and just continues
\end{itemize}

We used semaphores in two different ways:

\begin{itemize}
  \item The \textit{mutex} semaphore is used for mutual exclusion. It is designed to guarantee that only one process at a time will be reading or writing the buffer and the associated variables
  \item The other use of semaphores is for \textbf{synchronization}. The \textit{full} and \textit{empty} semaphores are needed to guarantee that certain event sequences do or do not occur.
\end{itemize}

\subsection*{Monitors.}

\begin{itemize}
  \item A \textbf{monitor} is a higher-level synchronization primitive - it is a collection of procedures, variables, and data structures that are all grouped together in a special kind of module or package
  \item Processes may call the procedures in a monitor, but they cannot access directly
\end{itemize}

\subsection*{Message Passing.}
\noindent

Is the method of IPC that uses two primitives, \textbf{send} and \textbf{receive}, which, like semaphores and unlike monitors, are system calls rather than language constructs

They can easily be put into library procedures.

\section*{Additional: TSL Instruction.}

\begin{itemize}
  \item Hardware-assisted approach; works with multiple CPUs
  \item An instruction like \textbf{TSL TX, LOCK} (Test and Set Lock) reads the contents of the memory work lock into register RX and then sets lock to be a nonzero value
  \item The operations of reading the word and storing into it are guaranteed to be indivisible
  \item Different from disabling interrupts - doing it on processor 1 has no effect at all on processor 2
\end{itemize}

\subsection*{More on mutexes.}

Can be realized with code primitives, do not need kernel functions, often have \textit{mutex\_lock} and \textit{mutex\_unlock} functions

If processes have disjoint address spaces how can they share the \textbf{turn} variable, semaphores or a common buffer? 

\begin{itemize}
  \item Some of the shared data structures, such as the semaphores, can be stored in the kernel and accessed only by means of system calls
  \item Most modern OSs offer a way for processes to share some portion of their address spaces with other processes. In this way, buffers and other data structures can be shared.
\end{itemize}

\subsection*{More on Monitors.}
\noindent

Problem with monitors, and also with semaphores, is that they were designed for solving the mutual exclusion problem on one or more CPUs that all \textbf{have access to a common memory}. In a distributed system consisting of multiple CPUs, each with its own private memory, these primitives become inapplicable.

\section*{Scheduling.}
\noindent

\begin{itemize}
  \item Batch systems: just run the next job on the tape 
  \item Multiprogramming systems: multiple users are waiting for service and CPU time is a scarce resource, so the scheduling algorithm became more complex 
  \item Personal computers: users are usually working in one application at a time and CPUs became much faster. As a consequence, scheduling does not matter much on simple PCs
  \item Network servers: multiple processes often do compete for the CPU, o scheduling matters again 
  \item Mobile devices: since battery lifetime is one of the most important constrains on these devices, some schedulers try to optimize the power consumption
\end{itemize}

\subsection*{When to schedule?}

\begin{itemize}
  \item when a new process is created 
  \item when a process exits
  \item when a process blocks on I/O or on a semaphore 
  \item when I/O interrupt occurs
\end{itemize}

A \textbf{nonpreemptive} scheduling algorithm do not make scheduling decisions during clock interrupts.
A \textbf{preemptive} scheduling algorithm picks a process and lets it run for a maximum of some fixed time. 

\subsection*{Goals.}

\begin{itemize}
  \item Fairness - giving each process a fair share of the CPU 
  \item Policy enforcement - seeing that stated policy is carried out 
  \item Balance - keeping all parts of the system busy. If the CPU and all the I/O devices can be kept running all the time, more work gets done per second than if some of the components are idle
\end{itemize}

\subsection*{Scheduling in Batch Systems.}

\begin{itemize}
  \item First-Come First-Served 
    \begin{itemize}
      \item processes are assigned the CPU in the order they request it
      \item when the jobs come in, they are put int the end of the queue
      \item when the running process blocks, the first process in the queue is run next 
      \item when a blocks process becomes ready, like a newly arrived job, it is put in the end of the queue, behind all waiting processes
    \end{itemize}
\end{itemize}


\end{document}
