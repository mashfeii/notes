\documentclass{article}
\usepackage[english]{babel}
\usepackage[letterpaper,top=2cm,bottom=2cm,left=2.5cm,right=2.5cm,marginparwidth=1.25cm]{geometry}

\usepackage[leqno]{amsmath}
\usepackage{enumitem, nccmath,lipsum,amssymb,array}

\newcommand{\mytext}[1]% #1 = same as intertext
{&\parbox{0.9\textwidth}{\rule{0pt}{.5\baselineskip}\\
\textrm{#1}\\
\rule{0pt}{.5\baselineskip}}&\\}

\newcounter{exercise}
\newcounter{problem}[exercise]
\newcommand{\myitem}{\stepcounter{problem}\tag*{\alph{problem})}}

\title{Probability and Statistics: Gmurman book}
\author{Mashenkov Timofei}
\begin{document}
\maketitle{}

\section*{Function of single variable}

If X - discrete random variable and function $Y(x) = \phi{X}$ is monotonic, then for $X$ values there are corresponding different $Y$ values, and the probabilities of $Y$ values are equal to the probabilities of $X$ values.

\begin{equation*}
  y_i = \phi(x_i)
\end{equation*}

\begin{equation*}
  P(Y = y_i) = P(X = x_i)
\end{equation*}

If $X$ is continuous random variable, and $y=\phi{x}$ - differentiable increasing or decreasing function:

\begin{eqnarray*}
  g(y) = f [\phi(y)] \cdot |\phi'(y)|
\end{eqnarray*}

\subsection*{Task 373.}

\begin{fleqn}[1\parindent]
  \begin{gather*}
    y_1 = 3\cdot 1 = 3;\ y_2 = 9;\ y_3 = 15 \\ 
    \begin{tabular}{c c c c}
      Y & 3 & 9 & 15 \\ 
      p & 0.4 & 0.1 & 0.5
    \end{tabular}
  \end{gather*}
\end{fleqn}

\subsection*{Task 386.}

\begin{fleqn}[1\parindent]
  \begin{gather*}
    f(x) = frac{1}{2\pi-0}=\frac{1}{2\pi} \\
    Y=\cos{X} \rightarrow \phi(y) = \arccos{y} \\ 
    g(y) = f[\phi_1(y)]\cdot|\phi_1'(y)|+[\phi_2(y)]+|\phi_2'(y)| \\
    \phi_1'(y)=(\arccos(y))'=-\frac{1}{\sqrt{1-y^2}} \\
    \phi_2'(y)=(\arccos(y))'=-\frac{1}{\sqrt{1-y^2}} \\
    |\phi_1'(y)|=|\phi_2(y)'|=\frac{1}{\sqrt{1-y^2}} \\ 
    g(y) = \frac{1}{2\pi}\frac{1}{\sqrt{1-y^2}}+\frac{1}{2\pi}\frac{1}{\sqrt{1-y^2}}=\frac{1}{\pi\sqrt{1-y^2}}
  \end{gather*}
\end{fleqn}

\subsection*{Task 388.}

\begin{fleqn}[1\parindent]
  \begin{gather*}
    f(x)=\frac{1}{\sigma\sqrt{2\pi}}e^{-\frac{x-\mu}{\sigma}^2} \\
  \end{gather*}
\end{fleqn}

\section*{Functions of two random variables}

If to each pair of possible values of random variables X and Y corresponds single possible value of random variable Z,
then Z is \textit{function of two random variables X and Y}.

\begin{equation*}
  Z = \phi(X,Y)
\end{equation*}

\end{document}
