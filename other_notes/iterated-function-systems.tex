\documentclass{article}
\usepackage[english]{babel}
\usepackage[letterpaper,top=2cm,bottom=2cm,left=2.5cm,right=2.5cm,marginparwidth=1.25cm]{geometry}

\usepackage[leqno]{amsmath}
\usepackage{enumitem, nccmath,lipsum,amssymb,xcolor,xparse,listings, blindtext}
\usepackage[most]{tcolorbox}

\NewDocumentCommand{\codeword}{v}{
\texttt{\textcolor{blue}{#1}}
}

\lstset{language=C, keywordstyle={\bfseries \color{blue}}}

\NewDocumentCommand{\mynote}{+O{}+m}{%
  \begingroup
  \tcbset{%
    noteshift/.store in=\mynote@shift,
    noteshift=1.5cm
  }
  \begin{tcolorbox}[nobeforeafter,
    enhanced,
    sharp corners,
    toprule=0.5pt,
    bottomrule=0.5pt,
    leftrule=0pt,
    rightrule=0pt,
    colback=green!10,
    #1,
    left skip=\mynote@shift,
    right skip=\mynote@shift,
    overlay={\node[right] (mynotenode) at ([xshift=-\mynote@shift]frame.west) {\textbf{Note:}} ;},
    ]
    #2
  \end{tcolorbox}
  \endgroup
  }
\makeatother

\newcommand{\mytext}[1]% #1 = same as intertext
{&\parbox{0.9\textwidth}{\rule{0pt}{.5\baselineskip}\\
\textrm{#1}\\
\rule{0pt}{.5\baselineskip}}&\\}

\newcounter{exercise}
\newcounter{problem}[exercise]
\newcommand{\myitem}{\stepcounter{problem}\tag*{\alph{problem})}}

\title{Iterated Function Systems (IFS). Chaos Games.}
\author{Mashenkov Timofei}
\begin{document}
\maketitle{}

\section*{Introduction. Koch's Curve.}

Fractal steel do not have the definition, but should implement several properties:

\begin{itemize}
  \item It is self-similar.
  \item Belongs to float size set.
    \begin{itemize}
      \item To one value of x there are several values of y. (ambiguity)
    \end{itemize}
\end{itemize}

Also, the property of Koch's curve is that it is growing on each iteration. For the simplest example with unit line it
is the following: 1, $4/3$, $(\frac{4}{3})^2$, $\dots$

\begin{equation*}
  L = \lim_{n \to \infty} \left(\frac{4}{3}\right)^n \to \infty
\end{equation*}

\section*{L-Systems}

\begin{center}
  \begin{tabular}{ |c|c|c| }
    \hline 
    Python command & Symbol & Description \\
    \hline 
    forward(x) & F & Move forward by x units \\
    backward(x) & B & Move backward by x units \\
    left(x) & L & Turn left by x degrees \\
    right(x) & R & Turn right by x degrees \\
    \hline
  \end{tabular} 
\end{center}

So the simplest curve generating will be the following sequence:

\begin{center}
  \begin{tabular}{|c|c|c|c|c|c|c|c|}
    \hline
    F & L & F & R & R & F & L & F \\
    \hline
  \end{tabular} 
\end{center}

To get the fractal, we need to repeat this sequence several times and take \textbf{the rule:}
Each symbol \textit{F} on new iteration will be replaced by the sequence itself.

\section*{Fractal dimension.}

\begin{equation}
  N \cdot r^d = 1
\end{equation}

\begin{equation}
  \begin{aligned}
    \log{(N \cdot r^d)} = \log{1} \\ 
    \log{N} + d \cdot \log{r} = 0 \\ 
    d \cdot \log{r} = -\log{N} \\
    -d \cdot \log{(1/r)}  -\log{N} \\ 
    d = \frac{\log{N}}{\log{1/r}}
  \end{aligned}
\end{equation}

\section*{Iterated Function Systems.}

At each iteration we apply several compressive functions on the set of points.

\begin{equation}
  E_1 = T_1(E_0)\lor T_2(E_0) \lor T_3(E_0)
\end{equation}

\begin{equation}
  T_1(\begin{bmatrix}
    x\\y
  \end{bmatrix}) = \begin{bmatrix}
  0.5&0\\0&0.5
  \end{bmatrix}\cdot\begin{bmatrix}
   x\\y 
  \end{bmatrix}
  +
  \begin{bmatrix}
    0 \\ 0
  \end{bmatrix}
\end{equation}

\end{document}
