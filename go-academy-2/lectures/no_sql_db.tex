\documentclass{article}

\usepackage{titlesec}
\titlelabel{\thetitle.\quad}

\usepackage[T1]{fontenc}
\usepackage{inconsolata}

\usepackage[english]{babel}
\usepackage[letterpaper,top=2cm,bottom=2cm,left=2cm,right=2cm,marginparwidth=1.25cm]{geometry}

\usepackage{hyperref, booktabs, float}
\usepackage[leqno]{amsmath}
\usepackage{enumitem, nccmath,lipsum,amssymb,xcolor,xparse,listings, blindtext}
\usepackage[most]{tcolorbox}

\setlength{\parindent}{6pt}

\definecolor{codegreen}{rgb}{0,0.6,0}
\definecolor{codegray}{rgb}{0.5,0.5,0.5}
\definecolor{codepurple}{rgb}{0.58,0,0.82}
\definecolor{backcolour}{rgb}{0.95,0.95,0.92}

\lstdefinestyle{mystyle}{
    backgroundcolor=\color{backcolour},   
    commentstyle=\color{codegreen},
    keywordstyle=\color{magenta},
    numberstyle=\tiny\color{codegray},
    stringstyle=\color{codepurple},
    basicstyle=\ttfamily\footnotesize,
    breakatwhitespace=false,         
    breaklines=true,                 
    captionpos=b,                    
    keepspaces=true,                 
    numbers=left,                    
    numbersep=5pt,                  
    showspaces=false,                
    showstringspaces=false,
    showtabs=false,                  
    tabsize=2
}

\lstset{style=mystyle}

\usepackage{graphicx}
\graphicspath{ {./attachments/} }

\definecolor{light-gray}{gray}{0.9}
\newcommand{\code}[1]{\colorbox{light-gray}{\texttt{#1}}}

\title{NoSQL Databases}
\author{Mashenkov Timofei \\ \href{t.me/mashfeii}{mashfeii}}
\begin{document}
\maketitle{}

\section{Types of Databases}
\noindent

\begin{itemize}
  \item \textbf{NoSQL}
    \begin{itemize}
      \item Document Stores
      \item Key-Value Stores
      \item Column-Family Stores
      \item Graph Databases
    \end{itemize}
  \item \textbf{NewSQL}
    \begin{enumerate}
      \item Google Spanner
      \item CockroachDB
      \item VoltDB
    \end{enumerate}
  \item \textbf{Graph}
    \begin{itemize}
      \item Neo4j
      \item JanusGraph
    \end{itemize}
  \item \textbf{Time Series}
    \begin{itemize}
      \item InfluxDB
      \item TimescaleDB
      \item OpenTSDB
    \end{itemize}
  \item \textbf{In-Memory}
    \begin{itemize}
      \item Redis
      \item Memcached
    \end{itemize}
  \item \textbf{Object Storage} 
    \begin{itemize}
      \item Amazon S3
      \item Google Cloud Storage
    \end{itemize}
\end{itemize}

\subsection{NoSQL Databases}
\noindent

\subsubsection{Document Stores}
\noindent

\begin{itemize}
  \item \textbf{MongoDB} - Document Store
  \item \textbf{CouchDB} - Stores data as JSON documents and supports RESTful API
  \item \textbf{ArangoDB} - Multi-model database
  \item \textbf{RethinkDB} - Real-time database, pushes data updates to connected clients
\end{itemize}

\subsubsection{Column-Family Stores}
\noindent

\begin{itemize}
  \item \textbf{Apache Cassandra} - Distributed database system
  \item \textbf{Apache HBase} - Open-source implementation of Google BigTable (almost unused nowadays)
  \item \textbf{ScyllaDB} - High-performance NoSQL database
\end{itemize}

\section{Why are there so many types of databases?}
\noindent

Let's try as an example build a simple social network, initially we contain the following entities:

\begin{itemize}
  \item Users
  \item Posts
  \item Comments
\end{itemize}

\subsection{SQL Databases: PostgreSQL}
\noindent

\colorbox{green}{\textbf{Advantages:}}

\begin{itemize}
  \item The data is normalized 
  \item The application is working
  \item Fixes are made with standard DB toolkit
\end{itemize}

\colorbox{red}{\textbf{Disadvantages:}}

\begin{itemize}
  \item To get the data we had to write a lot of SQL queries
  \item Since it gets a lot of data, it gets very slow
  \item Front-end fixes are done with hands and often with downtime
\end{itemize}

\subsubsection{NoSQL Databases: MongoDB}
\noindent

\colorbox{green}{\textbf{Advantages:}}

\begin{itemize}
  \item There is no complex joins
  \item Flexible schema - we can add whatever we want
  \item Faster reading
  \item Horizontal scaling
\end{itemize}

\colorbox{red}{\textbf{Disadvantages:}}

\begin{itemize}
  \item No ACID transactions
  \item Complex connections are hard and expensive to make
\end{itemize}

\subsubsection{How to Split the Data into Collections?}
\noindent

\textbf{Nested:} 

\begin{itemize}
  \item Data goes together in requests
  \item Object always lives in the parent's context
  \item Nested objects weight limited
\end{itemize}

\textbf{Linked/Separate:}

\begin{itemize}
  \item Data is used separately
  \item Object could have many links to other objects
  \item Avoid data duplication
\end{itemize}

\section{Analytics}
\noindent

\subsection{Analytical Queries in Production}
\noindent

\colorbox{green}{\textbf{Advantages:}}

\begin{itemize}
  \item Data is always up-to-date
  \item Just write queries
\end{itemize}

\colorbox{red}{\textbf{Disadvantages:}}

\begin{itemize}
  \item Production is getting slow and can be down
  \item There are laws regulating the user data
\end{itemize}

\subsection{OLTP - Online Transaction Processing}
\noindent

\begin{itemize}
  \item \textbf{Goal} - fast transaction processing 
  \item \textbf{Example} - web-shops, bank systems, CRM, ERP
  \item \textbf{Data Structure} - normalized (3NF) databases 
  \item \textbf{Data Volume} - small to medium
  \item \textbf{Frequency of Operations} - high (many small transactions per second)
  \item \textbf{Query Types} - simple \code{SELECT}, \code{INSERT}, \code{UPDATE}, \code{DELETE} 
  \item \textbf{Consistency} - high, since data should be consistent in real-time
\end{itemize}

\subsection{OLAP - Online Analytical Processing}
\noindent

\begin{itemize}
  \item \textbf{Goal} - data analysis, reports making, identifying trends/patterns
  \item \textbf{Example} - data warehouses, business intelligence systems
  \item \textbf{Data Structure} - denormalized schemas (star, snowflake) for fast access
  \item \textbf{Data Volume} - large
  \item \textbf{Frequency of Operations} - low (few complex queries per hour)
  \item \textbf{Query Types} - aggregations (\code{SUM}, \code{AVG}, \code{COUNT}), \code{GROUP BY}
  \item \textbf{Consistency} - eventual, since data is not required to be up-to-date
\end{itemize}

\section{Graph Databases}
\noindent

The main idea is to store connections between entities. For example, we need to store common friends between users.

\textbf{When to use?} 

\begin{itemize}
  \item You need to store complex connections between objects (one-to-many, many-to-many)
  \item If queries are connected with some depth (e.g. "Who is a friend of a friend through a friend?")
  \item If data is too heterogeneous (different types of objects and connections)
\end{itemize}

\textbf{Caution:} 

\begin{itemize}
  \item Avoid super-nodes (nodes with a lot of connections)
  \item Use directed connections
  \item Use right granularity (not too many connections) for spreading the data
\end{itemize}

\end{document}
